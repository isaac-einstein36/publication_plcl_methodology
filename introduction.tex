\section{Introduction}
\label{sec:introduction}

\subsection{Motivation}
\label{sec:introduction:motivation}
PLCL (poly(l-lactide-co-$\epsilon$-caprolactone)) is a copolymer of poly(lactide acid) (PLA) and poly($\epsilon$-caprolactone) (PCL). These polymers are commonly used implantable materials for their biodegradability, non-toxic degradation products, and biocompatibility. Polyester-based polymers have increasingly been researched for use in biomedical applications~\cite{RefWorks:RefID:303-luo2023fabrication,RefWorks:RefID:31-fernández2012synthesis,RefWorks:RefID:19-zhang2021synthesis,RefWorks:RefID:304-jeong2018mechanical}.

This research focused on extruding using a tabletop extruder. This methodology makes PLCL synthesis more accessible for small-scale labs. Additionally, it allows for fine-tuning of material composition which can directly impact degradation timeline, material properties, and performance characteristics (\hl{I forget what we meant by "performance characteristics"}).

\subsubsection{Increasing Accessibility}
\label{sec:introduction:motivation:increasingAccessibility}

Most existing methods for synthesizing PLCL require large, expensive equipment or specialized chemistry equipment and expertise (see Section~\ref{sec:introduction:existingMethods}).

This methodology uses a tabletop extruder, such as a 3Devo Filament Maker One, which requires relatively low overhead, raw material, and prior expertise to operate~\cite{RefWorks:RefID:392-3devo}. By decreasing these requirements, this methodology is accessible for laboratories with limited funding and space.

\subsubsection{Fine-Tuning Material Composition}
\label{sec:introduction:motivation:materialComposition}

Adjusting the material composition of PLCL, the percent composition of PLA and PCL co-polymers, can affect the degradation timeline, material properties, and performance characteristics of the output material.

\threesubsection{Degradation timeline}

The synthesis method as well as co-polymer material composition of PLCL affects the material's degradation timeline. By adjusting the breakdown of PLA and PCL in the blend, the time needed to fully biodegrade within the body can by altered~\cite{RefWorks:RefID:303-luo2023fabrication,RefWorks:RefID:507-kalita2020endoflife}.

\threesubsection{Material properties}

\threesubsection{Performance characteristics}

\hl{See Above Note}

\subsection{Existing Methods}
\label{sec:introduction:existingMethods}

\subsection{Knowledge Gap}
\label{sec:introduction:knowledgeGap}

% \subsection{First Subsection}


% \subsubsection{First Sub Subsection}


% \threesubsection{First lowest-level subsection}
